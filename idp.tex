% -*- fill-column: 80 -*-
%
% Modified from the Science website Individualized Development Plan (IDP)
% template.
\documentclass[answers,12pt]{exam}
\usepackage[margin=0.5in,top=0.8in,bottom=0.8in,headheight=14.5pt]{geometry}
\usepackage[hidelinks]{hyperref}
\usepackage{sectsty}            % section font size
\usepackage[inline]{enumitem}   % customize description list, inline
\usepackage{cleveref}           % \cref
\usepackage{xparse}             % \DeclareDocumentCommand
\usepackage{etoolbox}           % \ifblank

% Roman numeral sections
\renewcommand{\thesection}{\Roman{section}}
% Section font size
\sectionfont{\fontsize{12}{15}\selectfont}
% Disable printing "Solution:"
\renewcommand{\solutiontitle}{}
% Macro: Remove paragraph spacing
\newcommand{\unindent}{\vspace{-16pt}}
% Macro: text box
\DeclareDocumentCommand{\tbox}{ O{1.5in} m }{%
  \ifblank{#2}{%
    % Blank answer - use fixed height solutionbox
    \begin{solutionbox}{#1}%
      #2%
    \end{solutionbox}
  }{%
    % Filled answer - use variable height framed paragraph box
    \begin{framed}%
      #2%
    \end{framed}
  }
}

\title{Annual Progress Meeting - Core lab}
\date{2018}

\firstpageheadrule
\runningheadrule
\makeatletter
\lhead{\@title}
\rhead{\@date}
\cfoot{\thepage}
\makeatother

\begin{document}
\noindent
The following template provides some guiding questions that can facilitate an
annual career progress and mentoring meeting between the mentee and his/her
faculty mentor(s).

\begin{description}[leftmargin=0pt]
\item[Student / Postdoc:] complete \cref{committee,progress,wellbeing,goals} and
  attach your updated CV.  Provide both documents to your faculty mentor(s) in
  advance of scheduling your meeting.
\item[Student / Postdoc and Faculty Mentor(s):] discuss
  \cref{committee,progress,wellbeing,goals}, review goals and objectives and
  think of action steps towards progress.  Discuss with your mentor(s) and
  complete \cref{next} together: outlining action steps and activities you agree
  to do towards making progress and meeting stated goals and objectives.
\end{description}

\noindent\rule{\linewidth}{0.4pt}

\section{Committee formation}\label{committee}
Students should plan to form their committee prior to taking their perspective
by the summer after the second year at the latest.
\begin{questions}
\question
\filbreak List committee members \tbox{}
\question
\filbreak Have these committee members reviewed your specific aims for your
  perspective?  \tbox{}
\question
\filbreak Have you invited your committee member to your departmental seminar?
  \tbox{}
\end{questions}

\section{Overall progress: Review of the last year}\label{progress}
\begin{questions}
\question
\filbreak Highlight your major accomplishments in the past year (e.g., publications,
patents, honors or awards, grants or fellowships): \tbox[2in]{}
\question
\filbreak Brief overview of research progress in the past year: \tbox[3in]{}
\question
\filbreak List any presentations at professional meetings or conferences outside of
  UConn (or those held on campus but included an outside audience)- \tbox{}
\question
\filbreak Coursework/Training: What courses, seminars, conferences, lab meetings,
  etc. do you participate in? Are they meeting your needs?  If not, what else
  would be helpful?  \tbox{}
\question
\filbreak Teaching Activities: How much, in percent effort? Is this sufficient for
  developing multidisciplinary academic skills? In what ways could these
  activities be interfering with research productivity?  \tbox{}
\question
\filbreak Administrative and Other Duties, such as assistance with writing grants or
  mentoring graduate or undergraduate students: How much, in percent effort?
  Are these activities relevant to your development of academic or professional
  skills?  In what ways could these activities be interfering with your research
  productivity?  \tbox{}
\question
\filbreak Did you accomplish all that you agreed on doing with your mentor under the
  Action Plan during your initial meeting?  If no, what parts of the plan did
  you not accomplish and why? Describe/list any unusual or unanticipated
  challenges you experienced.  \tbox{}
\question Mentoring and Professional Collaborations:
  \begin{parts}
  \filbreak
  \part How often did you meet with your faculty mentor(s) last year?  \tbox{}
  \filbreak
  \part How would you rank the frequency of meetings?
    \begin{checkboxes}          % Replace one \choice with \correctchoice
    \choice Too few
    \choice Just right
    \choice Too many
    \end{checkboxes}
  \filbreak
  \part Who are your secondary mentors? Is that person a faculty member?  How
    often did you meet with them?  \tbox{}
  \filbreak
  \part Do you have collaborators outside of the lab?  Please list their names
    and roles.  \tbox{}
  \end{parts}
\end{questions}

\section{Wellbeing}\label{wellbeing}
\begin{questions}
\question
\filbreak Research environment: What features of the lab group or your relationships
  with colleagues and collaborators are most helpful and supportive to your
  wellbeing? What concerns could negatively affect your progress?  \tbox{}
\question
\filbreak Work-life balance: What do you do to maintain a balance between your work
  and life/personal needs?  What would you like to continue to do, or do
  differently next year?  \tbox{}
\question
\filbreak Do stresses or concerns exist in your personal life that could impact your
  work? How are things going generally? Are you able to take regular breaks and
  vacations?  \tbox{}
\end{questions}

\section{Goals and Objectives}\label{goals}
\begin{questions}
\question
\filbreak Refer to your previous discussion with your faculty mentor(s).  What
  changes or modifications took place?  \tbox{}
\question
\filbreak List up to 5 scientific and career objectives in the coming year.  \tbox{}
\question
\filbreak What opportunities at UConn and beyond can assist you in reaching your
  professional/non-scientific objectives?  For example, participation in
  meetings, courses or workshop attendance (identify meeting/workshop and date)?
  \tbox{}
\end{questions}

\noindent\rule{\linewidth}{0.4pt}

\noindent
To be developed jointly by the mentee and the mentor(s) during or after the
discussion

\section{Action Plan for your Next Steps}\label{next}
In carrying out activities that may assist you in meeting your
Research/Scientific objectives listed above-
\begin{questions}
\question
\filbreak Projected timeline for completing your current projects. \tbox{}
\question
\filbreak Projected timeline for your graduation.  \tbox{}
\question
\filbreak List the activities in which you and your mentor(s) agree you should
  participate that will support you in achieving your scientific and
  professional objectives in the coming year.  \tbox{}
\question
\filbreak What additional actions can you and your mentor continue to do, in order
  to help you be successful?  \tbox{}
\end{questions}

\end{document}
