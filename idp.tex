% -*- fill-column: 80 -*-
%
% Modified from the Science website Individualized Development Plan (IDP)
% template.
\documentclass[12pt]{article}
\usepackage[margin=0.5in,top=0.8in,bottom=0.8in,headheight=0pt]{geometry}
\usepackage{fancyhdr}
\usepackage[hidelinks]{hyperref}
\usepackage{sectsty}            % section font size
\usepackage[inline]{enumitem}   % customize description list, inline
\usepackage{amssymb}            % \boxtimes

% Roman numeral sections
\renewcommand{\thesection}{Part \Roman{section}}
% Section font size
\sectionfont{\fontsize{12}{15}\selectfont}
% Macro: Remove paragraph spacing
\newcommand{\unindent}{\vspace{-16pt}}
% Macro: text box
\newcommand{\tbox}{\\ \framebox[\linewidth][l]{\phantom{X}}}

\title{Annual Progress Meeting - Core lab}
\date{2018}

\pagestyle{fancy}
\makeatletter
\lhead{\@title}
\rhead{\@date}
\cfoot{\thepage}
\makeatother

\begin{document}
\noindent
The following template provides some guiding questions that can facilitate an
annual career progress and mentoring meeting between the mentee and his/her
faculty mentor(s).

\begin{description}[leftmargin=0pt]
\item[Student / Postdoc:] complete Parts I to III and attach your updated CV.
  Provide both documents to your faculty mentor(s) in advance of scheduling your
  meeting.
\item[Student / Postdoc and Faculty Mentor(s):] discuss Parts I to III, review
  goals and objectives and think of action steps towards progress.  Discuss with
  your mentor(s) and complete Part IV together: outlining action steps and
  activities you agree to do towards making progress and meeting stated goals
  and objectives.
\end{description}

\noindent\rule{\linewidth}{0.4pt}

\section{Committee formation}
Students should plan to form their committee prior to taking their perspective
by the summer after the second year at the latest.
\begin{enumerate}
\item List committee members \tbox{}
\item Have these committee members reviewed your specific aims for your
  perspective?  \tbox{}
\item Have you invited your committee member to your departmental seminar?
  \tbox{}
\end{enumerate}

\section{Overall progress: Review of the last year}
\begin{enumerate}
\item Highlight your major accomplishments in the past year (e.g., publications,
  patents, honors or awards, grants or fellowships): \tbox{}
\item Brief overview of research progress in the past year: \tbox{}
\item List any presentations at professional meetings or conferences outside of
  UConn (or those held on campus but included an outside audience)- \tbox{}
\item Coursework/Training: What courses, seminars, conferences, lab meetings,
  etc. do you participate in? Are they meeting your needs?  If not, what else
  would be helpful?  \tbox{}
\item Teaching Activities: How much, in percent effort? Is this sufficient for
  developing multidisciplinary academic skills? In what ways could these
  activities be interfering with research productivity?  \tbox{}
\item Administrative and Other Duties, such as assistance with writing grants or
  mentoring graduate or undergraduate students: How much, in percent effort?
  Are these activities relevant to your development of academic or professional
  skills?  In what ways could these activities be interfering with your research
  productivity?  \tbox{}
\item Did you accomplish all that you agreed on doing with your mentor under the
  Action Plan during your initial meeting?  If no, what parts of the plan did
  you not accomplish and why? Describe/list any unusual or unanticipated
  challenges you experienced.  \tbox{}
\item Mentoring and Professional Collaborations: a. How often did you meet with
  your faculty mentor(s) last year?
  \begin{enumerate}
  \item How often did you meet with your faculty mentor(s) last year?  \tbox{}
  \item How would you rank the frequency of meetings? \\
    \begin{itemize*}[label={}]
    \item $\square$ Too few
    \item $\boxtimes$ Just right
    \item $\square$ Too many
    \end{itemize*}
  \item Who are your secondary mentors? Is that person a faculty member?  How
    often did you meet with them?
  \item Do you have collaborators outside of the lab?  Please list their names
    and roles.  \tbox{}
  \end{enumerate}
\end{enumerate}

\section{Wellbeing}
\begin{enumerate}
\item Research environment: What features of the lab group or your relationships
  with colleagues and collaborators are most helpful and supportive to your
  wellbeing? What concerns could negatively affect your progress?  \tbox{}
\item Work-life balance: What do you do to maintain a balance between your work
  and life/personal needs?  What would you like to continue to do, or do
  differently next year?  \tbox{}
\item Do stresses or concerns exist in your personal life that could impact your
  work? How are things going generally? Are you able to take regular breaks and
  vacations?  \tbox{}
\end{enumerate}

\section{Goals and Objectives}
\begin{enumerate}
\item Refer to your previous discussion with your faculty mentor(s).  What
  changes or modifications took place?  \tbox{}
\item List up to 5 scientific and career objectives in the coming year.  \tbox{}
\item What opportunities at UConn and beyond can assist you in reaching your
  professional/non-scientific objectives?  For example, participation in
  meetings, courses or workshop attendance (identify meeting/workshop and date)?
  \tbox{}
\end{enumerate}

\noindent\rule{\linewidth}{0.4pt}

To be developed jointly by the mentee and the mentor(s) during or after the
discussion

\section{Action Plan for your Next Steps}
In carrying out activities that may assist you in meeting your
Research/Scientific objectives listed above-
\begin{enumerate}
\item Projected timeline for completing your current projects. \tbox{}
\item Projected timeline for your graduation.  \tbox{}
\item List the activities in which you and your mentor(s) agree you should
  participate that will support you in achieving your scientific and
  professional objectives in the coming year.  \tbox{}
\item What additional actions can you and your mentor continue to do, in order
  to help you be successful?  \tbox{}
\end{enumerate}

\end{document}
